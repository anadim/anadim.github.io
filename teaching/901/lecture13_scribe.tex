% -----------------------------*- LaTeX -*------------------------------
\documentclass[12pt]{report}
\usepackage{scribe}

\usepackage{amssymb,amsfonts,amsmath}
\usepackage{verbatim,comment}
\usepackage{enumitem}
%
\pagestyle{empty}
\def \bs{\boldsymbol}

\def \iid{\text{i.i.d.}}
\DeclareMathOperator*{\argmax}{arg\,max}
\DeclareMathOperator*{\argmin}{arg\,min}
\DeclareMathOperator*{\E}{\mathbb E}

\newcommand{\abs}[1]{\mid #1 \mid}
\newcommand{\set}[1]{\{#1\}}
\newcommand{\D}{\mathcal{D}}

\newtheorem{defn}[lemma]{Definition}
\newtheorem{question}[lemma]{Question}

\begin{document}
\scribe{Huayu Zhang}
\lecturenumber{13}			% required, must be a number
\lecturedate{10/18}		% required, omit year

\maketitle

\framebox[.95\textwidth]{\parbox{.93\textwidth}{ {{\bf Note:}} These
lecture notes are still rough, and have only have been mildly proofread.  }}
\vspace*{.1in}


% ----------------------------------------------------------------------

\section{Stability of learning algorithms}
In this lecture, we try to establish a connect between the stability and generalization of an algorithm. Specifically, algorithmic stability implies a good generalization error. Consider such a learning problem. A set of data $S = \set{Z_i = (\bs{x}_i, y_i) \mid i = 1,2,\ldots, n}$ are drawn independently from a distribution $Z_i \overset{\iid}{\sim}\D$. The model is characterized by parameters $\bs{w}$ and the loss function is $\ell(\bs{w};Z)$. 

\begin{defn}[Generalization error]
We define the risk as 
\begin{equation} \label{scr:risk}
	R(\bs{w}) = \E_{Z \sim \D}[\ell(\bs{w}; Z)]
\end{equation} 
and the empirical risk as
\begin{equation} \label{scr:emp-risk}
	\hat{R}(\bs{w}) = \frac{1}{n} \sum_{i=1}^{n} \ell(\bs{w};Z_i)
\end{equation}
The generalization error is the difference 
\begin{equation} \label{scr:generr}
	\epsilon = R(\bs{w}) - \hat{R}(\bs{w})
\end{equation}
\end{defn}

Let $A$ be the algorithm and $A(S)$ be the output model of the algorithm $A$ on data $S$. We can write
\begin{align*}
	R(A(S)) =& \E_{Z \sim \D}[\ell(A(S);Z)] \\
	\hat{R}(A(S)) =& \frac{1}{n} \sum_{i=1}^{n} \ell(A(S);Z_i) \\
	\E_S[\abs{\epsilon}] =& \E_S[\abs{R(A(S)) - \hat{R}(A(S))}]
\end{align*}

Next we define the stability of algorithms. The stability is a metric to show how the result of an algorithm varies with one data changed. On the data set $S$, replace one data point $Z_i$ with another i.i.d. random variable $Z_i'$. Then we get a new date set $S^i = (S \backslash \set{Z_i})\bigcup \set{Z_i'}$.
\begin{defn}[$\varepsilon$-Stable]
	Algorithm $A$ is $\varepsilon$-Stable if for any $i \in \set{1,2,\ldots,n}$, 
	\begin{equation} \label{scr:estable}
		\E_{S,Z_i',Z}[\abs{\ell(A(S_i);Z) - \ell(A(S^i);Z)}] \le \varepsilon
	\end{equation}
\end{defn}

The following theorem shows algorithmic stability implies good generalization error.
\begin{theorem}
	If $A$ is $\varepsilon$-stable, then $\E_S[\abs{R(A(S)) - \hat{R}(A(S))}] \le \varepsilon$.
\end{theorem} 

\begin{proof}
	\begin{align*}
		\E_S[\hat{R}(A(S))] =& \frac{1}{n} \sum_{i=1}^{n} \E_S[\ell(A(S);Z_i)] \\
		=& \frac{1}{n} \sum_{i=1}^{n} \{\E_{S,Z_i'}[\ell(A(S), Z_i')] + \E_{S,Z_i'}[\ell(A(S), Z_i) - \ell(A(S), Z_i')] \} \\
		\overset{\E_S[\ell(A(S),Z_i)] = \E_{S^i,Z_i'}[\ell(A(S^i),Z_i')]}{=}& \E_S[R(A(S))] + \frac{1}{n} \sum_{i=1}^{n} \E_{S,Z_i'}[\ell(A(S^i), Z_i') - \ell(A(S), Z_i')] \\
		\overset{A\text{ is $\varepsilon$-stable}}{\le} & \E_S[R(A(S))] + \varepsilon
	\end{align*}
\end{proof}

\begin{question}
	What algorithms are $\varepsilon$-stable?
\end{question}

\begin{enumerate}[label=$\bullet$]
	\item $A(S) = \argmin \sum_{i=1}^{n} \ell(\bs{w};Z_i)$. 
	\item $A(S) = $ the output of SGD after $T$ iterations.
\end{enumerate}

\begin{enumerate}[label=Case \Roman*]
	\item $A(S) = \argmin f_S(\bs{w}), f_S(\bs{w}) = L_S(\bs{w})  + \lambda \| \bs{w} \|^2, L_S(\bs{w}) = \frac{1}{n} \sum_{i=1}^{n} \ell_i(\bs{w})$. $\ell$ is $L$-Lipschitz. $A(S)$ is $O(\frac{1}{\lambda n})$-stable.
\end{enumerate}

\begin{proof}
	\begin{align*}
		f_S(\bs{V}) - f_S(\bs{u}) =& L_S(\bs{v}) + \lambda \| \bs{v} \|^2 - (L_S(\bs{u}) + \lambda \| \bs{u} \|^2) \\
		=& L_{S^i}(\bs{v}) + \lambda \| \bs{v} \|^2 - (L_{S^i}(\bs{u}) + \lambda \| \bs{u} \|^2) + \frac{\ell(\bs{v};z_i) - \ell(\bs{v};z_i')}{n} - \frac{\ell(\bs{u};z_i) - \ell(\bs{u};z_i')}{n} 
	\end{align*}
	Let $\bs{v} = A(S^i), \bs{u} = A(S)$, 
	\begin{align*}
		f_S(A(S^i)) - f_S(A(S)) \le& \frac{\ell(\bs{v};z_i) - \ell(\bs{v};z_i')}{n} - \frac{\ell(\bs{u};z_i) - \ell(\bs{u};z_i')}{n} \\
		\le& \frac{2L}{n} \| A(S) - A(S^i) \|
	\end{align*}
	According to the property of $2\lambda$-strongly convex function 
	\begin{equation*}
		f_S(\bs{w}) - f_S(A(S)) \ge \lambda \| \bs{w} - A(S) \|^2
	\end{equation*} 
	Thus
	\begin{equation} \label{scr:l2reg}
		\| A(S) - A(S^i) \| \le \frac{2L}{\lambda n}
	\end{equation}	
	By Eq.~\ref{scr:l2reg} and $L$-Lipschitz of $\ell_i$.
	\begin{equation*}
		\sup_Z \abs{\ell(A(S), Z) - \ell(A(S^i), Z)} \le \frac{2L^2}{\lambda n}
	\end{equation*}
	$A(S)$ is $\frac{2L^2}{\lambda n}$ stable.
\end{proof}
\end{document}

